\documentclass[]{article}

\usepackage[left=2.00cm, right=2.00cm, top=2.00cm, bottom=2.00cm]{geometry}
\usepackage[spanish,es-noshorthands]{babel}
\usepackage[utf8]{inputenc} % para tildes y ñ
\usepackage{graphicx} % para las figuras
\usepackage{xcolor}
\usepackage{listings} % para el código fuente en c++
\lstdefinestyle{customc}{
  belowcaptionskip=1\baselineskip,
  breaklines=true,
  frame=single,
  xleftmargin=\parindent,
  language=C++,
  showstringspaces=false,
  basicstyle=\footnotesize\ttfamily,
  keywordstyle=\bfseries\color{green!40!black},
  commentstyle=\itshape\color{gray!40!gray},
  identifierstyle=\color{black},
  stringstyle=\color{orange},
}
\lstset{style=customc}

%opening
\title{Práctica 4. Exploración de grafos}
\author{Carmen del Mar Ruiz de Celis \\ % mantenga las dos barras al final de la línea y este comentario
carmen.ruizdecelis@alum.uca.es \\ % mantenga las dos barras al final de la línea y este comentario
Teléfono: xxxxxxxx \\ % mantenga las dos barras al final de la linea y este comentario
NIF:49565250C  \\ % mantenga las dos barras al final de la línea y este comentario
}


\begin{document}

\maketitle

%\begin{abstract}
%\end{abstract}

% Ejemplo de ecuación a trozos
%
%$f(i,j)=\left\{ 
%  \begin{array}{lcr}
%      i + j & si & i < j \\ % caso 1
%      i + 7 & si & i = 1 \\ % caso 2
%      2 & si & i \geq j     % caso 3
%  \end{array}
%\right.$

\begin{enumerate}
\item Comente el funcionamiento del algoritmo y describa las estructuras necesarias para llevar a cabo su implementación.

$$ f(damage, attacksPerSecond, range, dispersion, health)= $$ $$(damage/attacksPerSecond)*4/5 + (range*dispersion*health)*1/5$$

El criterio de evaluación de las defensas consiste en una suma de dos partes, que se desglosa en:
\begin{itemize}
\item El daño por segundo, de la relación del daño entre los ataques por segundo, en el que recae la mayor parte de la nota al estar ponderado con un 4/5. 
Poco importa la salud, el rango o la dispersión si la defensa es superior en lo que respecta a daño.
\item El producto del rango, la dispersión y la salud. Claro está que el rango y la dispersión están estrechamente ligadas, pero, al tener la salud un gran papel 
ese 1/5, lo más correcto es utilizarlo como "potenciador" del resultado anterior (rango y dispersión).
\end{itemize}

\item Incluya a continuación el código fuente relevante del algoritmo.

La función de factibilidad comprueba si la defensa se puede colocar en una posición dadas la fila y la columna. Luego, es necesario comprobar si colisionaría con algún obstáculo u defensa y si no excedería los límites del mapa.



\end{enumerate}

Todo el material incluido en esta memoria y en los ficheros asociados es de mi autoría o ha sido facilitado por los profesores de la asignatura. Haciendo entrega de esta práctica confirmo que he leído la normativa de la asignatura, incluido el punto que respecta al uso de material no original.

\end{document}
