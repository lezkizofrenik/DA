El criterio que he elegido para la colocación de la primera defensa depende de dos factores: el número de obstáculos que haya alrededor y la distancia al centro del mapa. He considerado que el concepto ``alrededor'' es todo obstáculo a una distancia menor o igual al rango de alcance de la defensa.
La función devuelve el número de obstáculos menos la distancia entre el centro y la posición, ya que:
- La función de selección devuelve el mayor valor del mapa
- Nos interesa que se valore más que haya poca distancia respecto al centro
- La distancia siempre será mayor que el número de obstáculos

En este orden de los operandos, tendremos un número negativo, haciendo que la función de selección valore la poca cercanía y la cantidad de obstáculos

% Elimine los símbolos de tanto por ciento para descomentar las siguientes instrucciones e incluir una imagen en su respuesta. La mejor ubicación de la imagen será determinada por el compilador de Latex. No tiene por qué situarse a continuación en el fichero en formato pdf resultante.
%\begin{figure}
%\centering
%\includegraphics[width=0.7\linewidth]{./defenseValueCellsHead} % no es necesario especificar la extensión del archivo que contiene la imagen
%\caption{Estrategia devoradora para la mina}
%\label{fig:defenseValueCellsHead}
%\end{figure}
