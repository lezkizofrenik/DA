\begin{lstlisting}

void DEF_LIB_EXPORTED placeDefenses(bool** freeCells, int nCellsWidth, int nCellsHeight, float mapWidth, float mapHeight
              , std::list<Object*> obstacles, std::list<Defense*> defenses) {

   
    float cellWidth = mapWidth / nCellsWidth;
    float cellHeight = mapHeight / nCellsHeight; 
    int maxAttemps = 1000; //Número de intentos en el que debe encontrar una posición factible
   
    //inicializo el mapa de valores
    std::vector<std::vector<float>> MapValue(nCellsHeight, std::vector<float>(nCellsWidth)); //Mapa para el resto de defensas
    std::vector<std::vector<float>> MapValueFirstD(nCellsHeight, std::vector<float>(nCellsWidth)); //Mapa para para la primera

    std::list<Defense*> defensescopy = defenses; //Hago una copia de la lista de defensas para ordenarlas
    defensescopy.erase(defensescopy.begin()); //Borro la primera defensa
    defensescopy.sort(compare()); //Las ordeno según una puntuación de sus características

    // Le doy valores a los mapas
    for(int i = 0; i < nCellsHeight; i++){
        for(int j = 0; j < nCellsWidth; j++){
            MapValue[i][j] = cellValue(i, j, nCellsWidth, nCellsHeight, mapWidth, mapHeight, obstacles, *defenses.begin());
            MapValueFirstD[i][j] = FirstDefensecellValue(i, j, nCellsWidth, nCellsHeight, mapWidth, mapHeight, obstacles, defenses);
           
        }
    }
    
   int row, col; //variables para obtener la fila y la columna a partir de positiontoCell
   Vector3 v; //Guardará la posición devuelta por selection (la mejor del mapa)
   bool firstTime = false; //Variable que indicará si la primera defensa se ha colocado o no
   for( List<Defense*>::iterator currentDefense = defensescopy.begin(); currentDefense != defensescopy.end() && maxAttemps > 0; maxAttemps--){
        if(!firstTime){ //Como estoy recorriendo la copia sin la primera defensa, tengo que hacer este bloque aparte para no utilizar
                        //la lista original en el bucle for
            v = selection(MapValueFirstD, cellHeight, cellWidth, nCellsWidth, nCellsHeight, defenses); //obtengo la mejor posición del mapa
             positionToCell(v, row, col, cellWidth, cellHeight); //obtengo la fila y la columna de la mejor posición
            if(feasibility((*defenses.begin())->id, row, col, nCellsWidth, nCellsHeight, mapWidth, mapHeight, obstacles, defenses)){
                maxAttemps = 1000;
                (*defenses.begin())->position = v;
                firstTime=true;
                //Si la posición hayada por selection es factible, se resetean los intentos, se le asigna la posición a la defensa
                //y se pasa a la siguiente. En este bloque se hace lo mismo que en el del else, sólo que las funciones reciben
                //parámetros distintos (el mapa y el id de la primera defensa, que requiere de la lista original). Además, necesita
                //indicar que la primera defensa se ha colocado con la variable firstTime.
             } 
           
        }else{
          // En este bloque se hace lo mismo que arriba, sólo que 
            v = selection(MapValue, cellHeight, cellWidth, nCellsWidth, nCellsHeight, defenses);
            positionToCell(v, row, col, cellWidth, cellHeight);
            if(feasibility((*currentDefense)->id, row, col, nCellsWidth, nCellsHeight, mapWidth, mapHeight, obstacles, defenses)){
                maxAttemps = 1000;
                (*currentDefense)->position = v;
                currentDefense++;
            } 
        }
   }
   
}

struct compare {
    bool operator()(Defense * d1, Defense * d2) {return ((d1->health*1/5 + d1->attacksPerSecond*1/5 + d1->range*1/5 + d1->damage*1/5
                                                        + d1-> dispersion*1/5) > (d2->health*1/5 + d2->attacksPerSecond*1/5 + d2->range*1/5 + d2->damage*1/5
                                                        + d2-> dispersion*1/5));}
};

//Estructura usada como criterio de ordenación en la copia de la lista de defensas basándose en la puntuación de sus características.

\end{lstlisting}
