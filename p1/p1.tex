\documentclass[]{article}

\usepackage[left=2.00cm, right=2.00cm, top=2.00cm, bottom=2.00cm]{geometry}
\usepackage[spanish,es-noshorthands]{babel}
\usepackage[utf8]{inputenc} % para tildes y ñ
\usepackage{graphicx} % para las figuras
\usepackage{xcolor}
\usepackage{listings} % para el código fuente en c++

\lstdefinestyle{customc}{
  belowcaptionskip=1\baselineskip,
  breaklines=true,
  frame=single,
  xleftmargin=\parindent,
  language=C++,
  showstringspaces=false,
  basicstyle=\footnotesize\ttfamily,
  keywordstyle=\bfseries\color{green!40!black},
  commentstyle=\itshape\color{gray!40!gray},
  identifierstyle=\color{black},
  stringstyle=\color{orange},
}
\lstset{style=customc}


%opening
\title{Práctica 1. Algoritmos devoradores}
\author{Carmen del Mar Ruiz de Celis \\ % mantenga las dos barras al final de la línea y este comentario
carmen.ruizdecelis@alum.uca.es \\ % mantenga las dos barras al final de la línea y este comentario
Teléfono: xxxxxxxx \\ % mantenga las dos barras al final de la linea y este comentario
NIF:49565250C  \\ % mantenga las dos barras al final de la línea y este comentario
}


\begin{document}

\maketitle

%\begin{abstract}
%\end{abstract}

% Ejemplo de ecuación a trozos
%
%$f(i,j)=\left\{ 
%  \begin{array}{lcr}
%      i + j & si & i < j \\ % caso 1
%      i + 7 & si & i = 1 \\ % caso 2
%      2 & si & i \geq j     % caso 3
%  \end{array}
%\right.$

\begin{enumerate}
\item Describa a continuación la función diseñada para otorgar un determinado valor a cada una de las celdas del terreno de batalla para el caso del centro de extracción de minerales. 

$$ f(damage, attacksPerSecond, range, dispersion, health)= $$ $$(damage/attacksPerSecond)*4/5 + (range*dispersion*health)*1/5$$

El criterio de evaluación de las defensas consiste en una suma de dos partes, que se desglosa en:
\begin{itemize}
\item El daño por segundo, de la relación del daño entre los ataques por segundo, en el que recae la mayor parte de la nota al estar ponderado con un 4/5. 
Poco importa la salud, el rango o la dispersión si la defensa es superior en lo que respecta a daño.
\item El producto del rango, la dispersión y la salud. Claro está que el rango y la dispersión están estrechamente ligadas, pero, al tener la salud un gran papel 
ese 1/5, lo más correcto es utilizarlo como "potenciador" del resultado anterior (rango y dispersión).
\end{itemize}

\item Diseñe una función de factibilidad explicita y descríbala a continuación.

La función de factibilidad comprueba si la defensa se puede colocar en una posición dadas la fila y la columna. Luego, es necesario comprobar si colisionaría con algún obstáculo u defensa y si no excedería los límites del mapa.


\item A partir de las funciones definidas en los ejercicios anteriores diseñe un algoritmo voraz que resuelva el problema para el caso del centro de extracción de minerales. Incluya a continuación el código fuente relevante. 

\begin{lstlisting}
void selectDefenses(std::list<Defense*> defenses, unsigned int ases, std::list<int> &selectedIDs
            , float mapWidth, float mapHeight, std::list<Object*> obstacles){
    //Tengo en cuenta primero la primera defensa
    selectedIDs.push_front((*defenses.begin())->id);
    ases -= (*defenses.begin())->cost;
    defenses.pop_front(); // La elimino de la lista de defensas ya que debe quedar fuera de la búsqueda de la mejor combinación
    std::vector<std::vector<float>> m(defenses.size(), std::vector<float>(ases + 1)); //tabla de subproblemas
    
    cellvalue(m, defenses, ases); //relleno la tabla
    topDefenses(m, selectedIDs, defenses, ases); //interpreto la tabla y extraigo las mejores defensas

}

//algoritmo TSP
void cellvalue(std::vector<std::vector<float>> &m, std::list<Defense*> defenses, unsigned int ases){
    std::list<Defense*>::iterator it = defenses.begin();
    for(int i = 0; i < ases + 1; i++){
        if(i < (*it)->cost ) m[0][i] = 0;
        else m[0][i] = DefenseValue(*it);
    }

    for(int j =1; j < defenses.size(); j++, it++){
        for(int k = 0 ; k < ases + 1; k++){
            if(k < (*it)->cost) m[j][k] = m[j-1][k];
            else m[j][k] = std::max(m[j-1][k], m[j-1][k - (*it)->cost] + DefenseValue(*it));
        }
    }

}


\end{lstlisting}

\item Comente las características que lo identifican como perteneciente al esquema de los algoritmos voraces. 

\begin{lstlisting}
    
void generar(std::vector<int> &v){
    for(int i = 0; i < 15; i++) v.push_back(rand());
}

void comprobar(std::vector<int> &v){
     std::vector<int> ordenado = v;
     std::sort(ordenado.begin(), ordenado.end());
     if(std::equal(ordenado.begin(), ordenado.end(), v.begin())) 
        std::cout << "Ha ordenado correctamente" << std::endl;
     else std::cout << "Ha ordenado mal" << std::endl;
}

int main(){

    std::vector<int> v;
    std::cout << "QUICKSORT" << std::endl;
    generar(v);
    quickSort(v, 0, v.size()-1);
    comprobar(v);
    
    std::cout << "FUSION" << std::endl;
    generar(v);
    sortFusion(v, 0, v.size()-1);
    comprobar(v);

}

\end{lstlisting}

\item Describa a continuación la función diseñada para otorgar un determinado valor a cada una de las celdas del terreno de batalla para el caso del resto de defensas. Suponga que el valor otorgado a una celda no puede verse afectado por la colocación de una de estas defensas en el campo de batalla. Dicho de otra forma, no es posible modificar el valor otorgado a una celda una vez que se haya colocado una de estas defensas. Evidentemente, el valor de una celda sí que puede verse afectado por la ubicación del centro de extracción de minerales.

\begin{itemize}
\item En el algoritmo sin ordenación, la función de selección busca para cada defensa la mejor celda a la que se puede optar, y en el peor caso ésta sería la última. Luego, la complejidad sería, suponiendo que n es el número de defensas y m el número de celdas del mapa, $(n^m)$. 
\item Para la ordenación por fusión, la estructura se divide por la mitad, quedando dos partes de aproximadamente n/2. Ambas se ordenan de manera recursiva y luego han de combinarse, lo que requiere de n comparaciones. Luego, tendría una complejidad de t(n/2) + t(n/2) + n. En el caso de llegar a su tamaño mínimo, se realizaría una inserción directa, iterando n(n-1)/2 veces. Luego es de complejidad $(n^2)$.
\item En el de ordenación rápida, toma un elemento como pivote y reorganiza elsubvector para que a la izquierda del pivote queden los menores o iguales, y a su derecha los mayores. Luego, el peor caso se produce cuando el pivote está en un extremo, realizando así n(n-1)/2 iteraciones, siendo de complejidad $(n^2)$.
\item Según la propiedad de completitud y orden del montículo, el orden de ordenar es el equivalente al de inserción, que es de n log n (de acuerdo con la implementación del árbol que utiliza la librería de C++), ya que la altura del árbol es de log n y siempre inserta en el último nivel.   
\end{itemize}



\item A partir de las funciones definidas en los ejercicios anteriores diseñe un algoritmo voraz que resuelva el problema global. Este algoritmo puede estar formado por uno o dos algoritmos voraces independientes, ejecutados uno a continuación del otro. Incluya a continuación el código fuente relevante que no haya incluido ya como respuesta al ejercicio 3. 

\begin{lstlisting}
El algoritmo voraz, tanto para el centro de extracción como para el resto de defensas, se hacen conjuntamente en el mismo código, ya comentado en el ejercicio 3.
\end{lstlisting}


\end{enumerate}

Todo el material incluido en esta memoria y en los ficheros asociados es de mi autoría o ha sido facilitado por los profesores de la asignatura. Haciendo entrega de este documento confirmo que he leído la normativa de la asignatura, incluido el punto que respecta al uso de material no original.

\end{document}
